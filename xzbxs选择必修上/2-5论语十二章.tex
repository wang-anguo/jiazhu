\documentclass{zhvt-classic}

\title{高中语文上}[高~中~语~文~上]
\author{人民教育出版社}
\maker{壬寅夏潤州江南布衣重製}

\begin{document}

\maketitle

\grid{none}{all}
\insertgraphic[width=0.65\linewidth,angle=90]{zhuxi}
\clearpage

\gridall
\tableofcontents

\chapter*[]{论语十二章}[选自《论语译注》中华书局2006 年版]

子曰﹃君子食无求饱,居无求安,敏【勤勉】于事而慎
于言,就有道而正焉【到有道的人那里去匡正自己。有道,
指有才艺或有道德的人】,可谓好学也已。﹄ \hspace{1cm} 《学而》

子曰﹃人而【如果】不仁,如礼何【怎样对待礼呢】 ?人而不仁,如乐何?﹄ \hspace{1cm} 《八佾》


子曰﹃朝闻道,夕死可矣。﹄ \hspace{1cm} 《里仁》

子曰﹃君子喻【知晓,明白】于义,小人喻于利。﹄ \hspace{1cm} 《里仁》

子曰﹃见贤思齐焉,见不贤而内自省也。﹄ \hspace{1cm} 《里仁》

子曰﹃质胜文则野【质朴超过文采就会粗野鄙俗。质,质朴、朴实。文,华美、文采。野,粗野、鄙俗。】,
文胜质则史【虚饰,浮夸。】。文质彬彬【文质兼备、配合适当的样子。】,然后君子。﹄ \hspace{1cm} 《雍也》

曾子曰﹃士不可以不弘毅【 \hspace{0.5cm}〔 弘毅〕\hspace{0.5cm}志向远大,意志坚强。弘,广、大,这里指志向远大。】,
任重而道远。仁以为己任,不亦重乎?死而后已,不亦远乎?﹄ \hspace{1cm} 《泰伯》

子曰﹃譬如为山,未成一篑【篑音溃,只差一筐土没有成功。篑,盛土的竹筐。】,
止,吾止也【 \hspace{0.5cm}〔 止,吾止也〕\hspace{0.5cm}停下来,是我自己停下来的。】。
譬如平地【 \hspace{0.5cm}〔 平地〕\hspace{0.5cm}填平洼地。】,虽覆一篑,进,吾往也。﹄ \hspace{1cm} 《子罕》

子曰﹃知【 \hspace{0.5cm}〔 知〕\hspace{0.5cm}同﹃智﹄。】者不惑,仁者不忧,勇者不惧。﹄ \hspace{1cm} 《子罕》

颜渊问仁。子曰﹃克己复礼【 \hspace{0.5cm}〔克己复礼〕\hspace{0.5cm}约束自我,使言行归复于先王之礼。】为仁。
一日【 \hspace{0.5cm}〔 一日〕\hspace{0.5cm}一旦。】克己复礼,天下归【 \hspace{0.5cm}〔 归〕\hspace{0.5cm}称赞,称许。】仁焉。为仁由己,而由人乎哉?﹄

颜渊曰﹃请问其目【 \hspace{0.5cm}〔 目〕\hspace{0.5cm}条目,细则。】。﹄子曰﹃非礼勿视,非礼勿听,非礼勿言,非礼勿动。﹄

颜渊曰﹃回虽不敏,请事【 \hspace{0.5cm}〔 事〕\hspace{0.5cm}实践,从事。】斯语矣。﹄ \hspace{1cm} 《颜渊》

子贡问曰﹃有一言【 \hspace{0.5cm}〔 一言〕\hspace{0.5cm}一个字。】而可以终身行之者乎?﹄子曰﹃其‘恕’乎!己所不欲,勿施于人。﹄ \hspace{1cm} 《卫灵公》

子曰﹃小子【 \hspace{0.5cm}〔 小子〕\hspace{0.5cm}老师对学生的称呼。】
何莫学夫【 \hspace{0.5cm}〔 夫〕\hspace{0.5cm}那。】《 诗》?
《诗》可以兴【 \hspace{0.5cm}〔 兴〕\hspace{0.5cm}指激发人的感情。】,可以观【 \hspace{0.5cm}〔 观〕\hspace{0.5cm}指观察政治的得失、风俗的盛衰。】,
可以群【 \hspace{0.5cm}〔 群〕\hspace{0.5cm}指提高人际交往能力。】,可以怨【 \hspace{0.5cm}〔 怨〕\hspace{0.5cm}指讽刺时政。】。
迩【 \hspace{0.5cm}〔 迩音耳〕\hspace{0.5cm}近。】之事父,远之事君。多识于鸟兽草木之名。﹄ \hspace{1cm} 《阳货》



\end{document}
